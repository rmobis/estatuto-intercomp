{\let\clearpage\relax \chapter{Voleibol de Areia de Dupla}}

\begin{article}
	A modalidade será disputada na categoria absoluto.
\end{article}

\begin{article}
	Durante os jogos serão obedecidas as regras oficiais, ressalvando os dispostos nos demais artigos deste regulamento.
\end{article}

\begin{article}
	Cada participante deverá inscrever entre 2 (dois) e 3 (três) atletas por confronto de xadrez, devendo indicar, antes da realização dos emparceiramentos, uma atleta como Atleta A, um atleta como Atleta B e um atleta como Atleta C, se houver.
\end{article}

\begin{article}
	Durante a \textit{Fase Classificatória}, no sistema de \textit{Disputa em Grupos}, Cada partida de xadrez terá duração máxima de 10 (dez) minutos por jogador
\end{article}

\begin{article}
	Cada confronto de xadrez será disputada em 3 (três) partidas, com os Atletas A de cada participante jogando entre si, os Atletas B de cada participante jogando entre si e os Atletas C de cada participante jogando entre si, se houverem.

	\begin{xparagraph}
		A cada partida, é atribuída a seguinte pontuação para as participantes: 1 (um) ponto para partida vencida, 1/2 (meio) ponto para partida empatada e 0 (zero) pontos para partida perdida.
	\end{xparagraph}

	\begin{xparagraph}
		Considera-se vencedora do confronto a participante com mais pontos.
	\end{xparagraph}

	\begin{xparagraph}
		Caso algum jogador não tenha comparecido a o início do confronto de xadrez, será acionado o relógio daquele jogador no horário marcado para início da partida.
	\end{xparagraph}
\end{article}

\begin{article}
	Fica definido para o Xadrez o seguinte \textit{Sistema de Classificação}, determinando os seguintes critérios, em ordem de importância, para ordenar as participantes após a \textit{Fase Classificatória}, no sistema de\textit {Disputa em Grupos}:

	\begin{xparagraph}
		Proporção de pontos conquistados dentre pontos disputados.
	\end{xparagraph}

	\begin{xparagraph}
		Número total de confrontos vencidos.
	\end{xparagraph}

	\begin{xparagraph}
		Número total de confrontos empatados.
	\end{xparagraph}

	\begin{xparagraph}
		Persistindo o empate, cada participante deverá indicar um atleta para disputa de uma partida relâmpago de desempate, com duração máxima de 5 (cinco) minutos por jogador.
	\end{xparagraph}
\end{article}

\begin{article}
	Durante a \textit{Fase Final}, no sistema de \textit{Disputa em Eliminatórias Simples}, cada partida de xadrez terá duração máxima de 15 (quinze) minutos por jogador.
\end{article}

\begin{article}
	É de responsabilidade de cada equipe trazer, a cada confronto, 2 (dois) jogos de peças, 2 (dois) tabuleiros e 2 (dois) relógios de xadrez, em perfeito estado de funcionamento.

	\begin{xparagraph}
		Caso, no momento da realização do confronto, não houver material suficiente em condições de uso para o seu início, a entidade que tiver deixado de cumprir a orientação do caput deste artigo será punida com W.O. no confronto.
	\end{xparagraph}
\end{article}

\begin{article}
	Fica proibido o uso de instrumento sonoro no local de disputa do xadrez, sendo considerados distúrbios de torcida.

	\begin{xparagraph}
		Caso ocorram distúrbios de torcida, a arbitragem ou o representante da competição poderá solicitar a retirada das torcidas do local da competição.
	\end{xparagraph}
\end{article}
